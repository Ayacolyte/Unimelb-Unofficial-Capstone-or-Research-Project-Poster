%% Unimelb unofficial Capstone/Unimelb Poster Template
%% By Changyuan (David) Ni
%
% This work is modified from https://cn.overleaf.com/latex/templates/unofficial-template-for-hong-kong-university-of-science-and-technology/bkxykbtssfpb

% Required Compiler: LuaLaTeX
%[draft]
\documentclass{beamer}
%\usepackage{extsizes}
% ====================
% Packages
% ====================

\usepackage[T1]{fontenc}
\usepackage{lmodern}
\usepackage[size=custom,width=118.9,height=84.1,scale=1.2]{beamerposter} % default
%\usepackage[orientation=landscape,size=a0,scale=1.4]{beamerposter} % custom
% scale: scaling factor of all font

\usetheme{gemini}
\usecolortheme{Tuebingen} % Customize in beamercolorthemeTuebingen.sty
 
\usepackage{wrapfig2}
\usepackage{graphicx}
\usepackage{booktabs}
\usepackage{tikz}
\usepackage{lipsum}
\usepackage{duckuments}
\usepackage{pgfplots}
\pgfplotsset{compat=1.14}
\usepackage{anyfontsize}
\usepackage[RGB]{xcolor}
%\usepackage{floatrow}
%\usepackage{enumitem}
\definecolor{dred}{RGB}{209, 51, 59}


%\sethlcolor{lightorange}



% ====================
% Lengths
% ====================

% If you have N columns, choose \sepwidth and \colwidth such that
% (N+1)*\sepwidth + N*\colwidth = \paperwidth
\newlength{\sepwidth}
\newlength{\colwidth}
\setlength{\sepwidth}{0.025\paperwidth}
\setlength{\colwidth}{0.3\paperwidth}

\newcommand{\separatorcolumn}{\begin{column}{\sepwidth}\end{column}}

% ====================
% Title
% ====================
%
\title{Project Title \\ \normalsize Booth x, Project Code: a-bc-12-defg-345}

\author{Changyuan (David) Ni \inst{1}, Supervised by:\and A/prof John doe \inst{1} \and Dr. Jane Doe \inst{1}}

\institute[shortinst]{\inst{1} Department of Biomedical Engineering, University of Melbourne}

% ====================
% Footer (optional)
% ====================

\footercontent{
  
  Endeavour 2024, University of Melbourne \hfill
  \href{mailto:youremail}{xxxxx@student.unimelb.edu.au}}
% (can be left out to remove footer)

% ====================
% Logo (optional)
% ====================

% use this to include logos on the left and/or right side of the header:
%\logoright{\includegraphics[height=4cm]{UST_MBA_Logo_White.png}}
%\logoleft{\includegraphics[height=7cm]{logo2.pdf}}

% ====================
% Body
% ====================

\begin{document}

\begin{frame}[t]
\begin{columns}[t]
\separatorcolumn

  
\begin{column}{\colwidth}
    
  \begin{block}{Background}
    \textcolor{dred}{\textbf{Some Topic}}\\
    \lipsum[1][1-3]
    \setlength{\columnsep}{1.5cm} % Adjust column separation
    \begin{wrapfigure}[10]{r}[10pt]{0.52\textwidth}  
        %\setlength{\columnsep}{10em}
        
        %\setlength{\wrapoverhang}{-1em} % Add negative space to the left
        \hfill
        
        \includegraphics[width=0.45\textwidth]{example-image-duck}
        \caption{Duck (figure adapted from duck et al., 2025)}
        \label{fig:your-label}
    \end{wrapfigure}
    %\vspace{-2em}
   
    

    \colorbox[RGB]{255, 245,242}{%
        \begin{minipage}{\linewidth}
        A highlight box here
        \begin{itemize}
            \item \textbf Some important statistics as dot points here
            \item \textbf Another important statistics that underpins your background
        \end{itemize}
        \end{minipage}}


        
    
    \lipsum[1][1-4]
    % \colorbox[RGB]{255, 245,242}{%
    %     \begin{minipage}{\linewidth}
        
    %     \end{minipage}}\\
    %\vspace{1em} % Adjust the space as needed
    
    \textcolor{dred}{\textbf{Topic 2}}\\
    \lipsum[1][1-3] 
    \begin{itemize}
        \item \textbf{dot point a:} \lipsum[1][1]
        \item \textbf{dot point b:} \lipsum[1][1]
        \item \textbf{dot point c:} \lipsum[1][1]    
    \end{itemize}
    

    \end{block}

    \begin{block}{Introduction and Aim}
        \setlength{\columnsep}{0.3cm} % Adjust column separation
        \begin{wrapfigure}[18]{l}{0.66\textwidth}  
        %\setlength{\columnsep}{10em}
        
            %\setlength{\wrapoverhang}{3em} % Add negative space to the left
            %\centering
            \includegraphics[width=0.66\textwidth]{example-image-duck}
            \caption{\lipsum[1][1-12]}
            \label{fig:your-label}
        \end{wrapfigure}
    \textcolor{dred}{\textbf{Motivation}}
    
    \lipsum[1][1-2]
    \begin{itemize}
        \small
        \setlength{\leftmargini}{-5em}
        \item point a
        \item point b 
    \end{itemize}
    \normalsize
    \textcolor{dred}{\textbf{Aim}}
    \colorbox[RGB]{255, 245,242}{%
        \begin{minipage}{350pt}
            \begin{itemize}
                \item \lipsum[1][1-2]
                \item \lipsum[1][1-2]
            \end{itemize}
        \end{minipage}}

    \end{block}
    


    
\end{column}

\separatorcolumn

\begin{column}{\colwidth}
    
    \begin{block}{Method}

    \lipsum[1][1-3]
    \begin{figure}
        \begin{minipage}[c]{0.83\textwidth}
        \includegraphics[width=\textwidth]{example-image-duck}
        \end{minipage}\hfill
        \begin{minipage}[c]{0.17\textwidth}
        \caption{\lipsum[1][1-4]}
        \label{fig:03-03}
        \end{minipage}
    \end{figure}
    % \begin{wrapfigure}[26]{l}{0.7\textwidth}  
    %     %\setlength{\columnsep}{10em}
        
    %         %\setlength{\wrapoverhang}{-3em} % Add negative space to the left
    %         %\centering
    %         \includegraphics[width=0.7\textwidth]{Endeavour poster/NN.png}
    %         \caption{\textbf{Top: }First layer is convolutional with max pooling + linear activation, second layer is fixed weights + sigmoid activation and third layer is FC + ReLU. \textbf{Bottom: }First layer is FC + linear activation, second layer is fixed weights + sigmoid activation and third layer is FC + ReLU}
    %         \label{fig:your-label}
    %     \end{wrapfigure}
    %     \par
    \lipsum[1][1]
    \begin{itemize}
        \item main point 1
        \begin{itemize}
            \item sub point 1
            \item sub point 2
        \end{itemize}
        \item main point 2
        \item main point 3
    \end{itemize}
     \lipsum[1][1-5]
    \end{block}
    \begin{block}{Results}
    \begin{figure}
        \begin{minipage}[c]{0.83\textwidth}
        \includegraphics[width=\textwidth]{example-image-duck}
        \end{minipage}\hfill
        \begin{minipage}[c]{0.17\textwidth}
        \caption{
       \lipsum[1][1-3] }
       \label{fig:03-03}
        \end{minipage}
    \end{figure}
    \lipsum[1][1-4]



    \end{block}


\end{column}

\separatorcolumn

\begin{column}{\colwidth}

  \begin{block}{Results (continued)}
     
    
    % \setlength{\columnsep}{1.5cm} % Adjust column separation
    % \begin{wrapfigure}[10]{r}[10pt]{0.52\textwidth}  
    %     %\setlength{\columnsep}{10em}
        
    %     %\setlength{\wrapoverhang}{-1em} % Add negative space to the left
    %     \hfill
    %     \includegraphics[width=0.45\textwidth]{Endeavour poster/Reconstruction comparison.png}
    %     \caption{Snippet of electrode pattern, neural activation pattern and reconstruction}
    %     \label{fig:your-label}
    % \end{wrapfigure}
    % \begin{wrapfigure}[10]{l}[10pt]{0.52\textwidth}  
    %     %\setlength{\columnsep}{10em}
        
    %     %\setlength{\wrapoverhang}{-1em} % Add negative space to the left
    %     \hfill
    %     \includegraphics[width=0.45\textwidth]{Endeavour poster/RF.png}
    %     \caption{Snippet of electrode pattern, neural activation pattern and reconstruction}
    %     \label{fig:your-label}
    % \end{wrapfigure}
    \begin{figure}
        \begin{minipage}[c]{0.5\textwidth}
            % This section shows snippets of model output and intermediate layer expression. 1 of 10000 validation images were shown as an example.\\~\\~\\~\\
            \includegraphics[width=\textwidth]{example-image-duck}
            \includegraphics[width=\textwidth]{example-image-duck}
            
            \caption{\lipsum[1][1-3]}
            
            \colorbox[RGB]{255, 245,242}{%
            \begin{minipage}{460pt}

            \begin{itemize}
                \item \lipsum[1][1-2]
                \item \lipsum[1][1-2]
            \end{itemize}
            \end{minipage}}

        \end{minipage}\hfill
        \begin{minipage}[c]{0.5\textwidth}
            \includegraphics[width=\textwidth]{example-image-duck}
            \caption{\lipsum[1][1-4]}
            \includegraphics[width=\textwidth]{example-image-duck}
            \caption{\lipsum[1][1-4]}
        \end{minipage}
    \end{figure}
    
    \lipsum[1][1]
    \begin{itemize}
        \item \lipsum[1][1-3]
        \item \lipsum[1][1-3]
    \end{itemize}
    
    

  \end{block}
  \begin{exampleblock}{Discussion and Future Work}
  \begin{itemize}
      \item \lipsum[1][1]
      \item \lipsum[1][1]
      
      \item \textbf{Next step:} 
      \begin{itemize}
          \item \lipsum[1][1-2]
          \item \lipsum[1][1-2]
      \end{itemize}
  \end{itemize}
  \end{exampleblock}
    


\end{column}
\end{columns}

\end{frame}

\end{document}
